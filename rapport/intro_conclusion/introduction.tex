
\pagebreak

\section*{Introduction}
\addcontentsline{toc}{section}{Introduction}

L'exercice proposé a pour finalité la modélisation et l'analyse d'un système ATC simplifié. En décomposant un système en fonctions, composants et flux, l'approche MBSE permet notamment de vérifier que toutes les fonctions sont associées à leur composant pertinent. Des corrections peuvent ainsi intervenir très tôt dans le cycle en V.

Le système ATC considéré est constitué d'acteurs humains, les contrôleurs, et d'acteurs systèmes
hardware et software. On citera les radars primaires, secondaires, les logiciels de traitement, de transport et d'affichage des données radar et plan de vol sans oublier les filets de sauvegarde. Le NMOC européen, Network Management Operation Centre, et le système de traitement de plan de vol national sont inclus dans le système étudié car ils participent à la fourniture du service de contrôle.

Les acteurs extérieurs sont en particulier les pilotes, les compagnies aériennes, la référence horaire GPS, les aéronefs+transpondeur, les services météo. On pourrait ajouter les acteurs humains impactant la sûreté et la sécurité du système directement par le hacking du réseau ATC ou par brouillage des communications radio, par détournement de vols, etc. Ceci impacte sur les contraintes du système sans oublier l'acteur environnemental des phénomènes météo. 







